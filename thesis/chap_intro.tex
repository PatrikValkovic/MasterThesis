\chapter{Introduction}
\label{chap:intro}

\acrfull{acc:ai} has become a phenomenon of these days. The latest advances in this field have achieved magnificent results and allowed the creation of systems and tools, which we would not think were possible thirty years ago. The biggest credit goes to the \acrfull{acc:ann}, which allowed this rapid and astonishing growth of the field over the past years. More precisely, the \acrfull{acc:dann} sparkled this process in 2011, when they started to exceed human performance on German traffic sign recognition benchmark \citep{CIRESAN2012333}.

Even though there had already been known industrial applications of AI at the beginning of the 2000s, such as automatic check processing \citep{ChecksDocumentRecognition} or speaker recognition \citep{HECK2000181}, it were mainly achievements in 2011 and 2012 which acquired attention from academia and the broader community. The progress in \acrshort{acc:ai} is moving forward day--by--day, and new results can be seen almost on a daily basis. The current state-of-the-art systems have already exceed human performance and traditional methods in a number of various tasks -- like 
image recognition \citep{pham2021meta}\citep{ZawadzkaGosk2019},
object detection \citep{ghiasi2020simple}\citep{lehner2019patch},
video editing \citep{lu2020layered},
question answering \citep{zhang2020pushing}\citep{yamada2020luke},
natural language processing \citep{gpt3},
and many more. Thanks to the rise of generative models, results in the field of
super-resolution \citep{Sun_2020}\allowbreak\citep{Chadha_2020},
image synthesis \citep{StateOfTheArtImageSythesis}\allowbreak\citep{esser2020taming}\allowbreak\citep{dalle},
text generation \citep{gpt3}\allowbreak\citep{malmi2019encode},
and countless more amaze scientists by their accuracy and attention to details.

\acrshort{acc:ann} also found their way into the fields which traditional machine learning methods have not considered -- fluid simulation \citep{um2018liquid}\citep{Kim_2019}, cloth simulation \citep{lee2019efficient}\citep{SRBO20}, or even whole physics \citep{PhysicsSimulation}\citep{sanchezgonzalez2020learning} and movement of virtual entities \citep{PhysicsBasedCharaterSImulation}\citep{zhang2020vid2player}. Because of that, \acrshort{acc:ann} are used in 
healthcare \citep{fakoor2013using}\allowbreak\citep{BreastCancerAISystem},
Hollywood \citep{aiinhollywood},
robotics \citep{pierson2017deep}\allowbreak\citep{Lee_2020},
and can even play computer games \citep{openai2019dota}\allowbreak\citep{alphastar}.

This long and incomplete list of successes would not be possible without the computation power of computers and data centers nowadays. In fact, the achievements of \acrshort{acc:dann} in 2011 and 2012 were driven mainly by the effective implementation of \acrfull{acc:cnn} in kernels and running them by the \acrfull{acc:gpu} \citep{CIRESAN2012333}. Since then, the \acrshort{acc:gpu} and more recently the \acrfull{acc:tpu} have overcome traditional processors in the terms of speed and performance for \acrshort{acc:ann}, and allowed these advancements in the field we can see now.

Another interesting field of \acrlong{acc:ai} is evolutionary computation. Similarly to \acrshort{acc:ann}, it performs parameters search in order to best solve the task at hand. Evolutionary algorithms and their variations have been successfully applied in a variety of fields -- for example 
electronic circuits design \citep{NASAantenaDesign}\allowbreak\citep{circuitdesignoptimizationea},
real-parameter optimization \citep{IntroNaturalEvolutionStrategies}, 
combinatorial problems \citep{GeneticAssambleLineBalancingProblem}\allowbreak\citep{ALBAYRAK20111313},\linebreak
robotics \citep{EvolutionaryRobotics}\allowbreak\citep{RoboticsInPhysX}\allowbreak\citep{nygaard2018realworld},
design of neural networks \citep{NEAT}\allowbreak\citep{Floreano2008NeuroevolutionFA},
and many more.

There have been attempts to implement and run evolutionary algorithms on \acrshort{acc:gpu} \citep{CHENG2019514} and use the computation power available. However, most of these methods focus on \acrfull{acc:cuda}. \acrshort{acc:cuda} and similar technologies have drawbacks -- they use low-end languages like C or \cpp and, therefore, require deep knowledge of the underlying hardware. The algorithms are also harder to modify.

Therefore, the aim of this work is to analyze existing implementations of evolutionary algorithms and propose their implementation on the \acrshort{acc:gpu}. The implementation needs to be easy to understand, extendable, and easily to modify. In order to meet the given criteria, implementation should reuse existing frameworks available for neural networks if possible.

The rest of this work is organized as follows. 
In the next Chapter, the evolutionary algorithms and their variants are introduced. 
Chapter~\ref{chap:gpu} describes \acrshort{acc:cuda} programming in order to  design efficient implementation.
Proposed implementation and design decisions are given in Chapter~\ref{chap:impl}. 
In Chapter~\ref{chap:problems}, set of problems is presented for the purpose of test, comparison, and benchmark of the implementation. This is followed by the evaluation of the proposed implementation and its comparison to the standard methods.
Finally, conclusion and main ideas are given in Chapter~\ref{chap:conclusion}.
